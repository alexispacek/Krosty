\documentclass[11pt,spanish]{article}
\usepackage[utf8]{inputenc}
\usepackage{float}
\usepackage{siunitx}

%Packages del style de Nico
\usepackage[T1]{fontenc}
\usepackage[spanish]{babel}
\usepackage{geometry}
\geometry{verbose,tmargin=2cm,bmargin=2cm,lmargin=2cm,rmargin=2cm}
\usepackage{amssymb}
\usepackage{amstext}
\usepackage{amsmath}
\usepackage{graphicx}
\usepackage{listings}

\usepackage{xcolor}

%New colors defined below
\definecolor{codegreen}{rgb}{0,0.6,0}
\definecolor{codegray}{rgb}{0.5,0.5,0.5}
\definecolor{codepurple}{rgb}{0.58,0,0.82}
\definecolor{backcolour}{rgb}{0.95,0.95,0.92}

%Code listing style named "mystyle"
\lstdefinestyle{mystyle}{
  backgroundcolor=\color{backcolour}, commentstyle=\color{codegreen},
  keywordstyle=\color{magenta},
  numberstyle=\tiny\color{codegray},
  stringstyle=\color{codepurple},
  basicstyle=\ttfamily\footnotesize,
  breakatwhitespace=false,         
  breaklines=true,                 
  captionpos=b,                    
  keepspaces=true,                 
  numbers=left,                    
  numbersep=5pt,                  
  showspaces=false,                
  showstringspaces=false,
  showtabs=false,                  
  tabsize=2
}

%"mystyle" code listing set
\lstset{style=mystyle}
%\usepackage[numbers]{natbib}
%\setlength{\bibsep}{0.0pt}
\usepackage[auth-sc]{authblk}
\usepackage{hyperref}
\usepackage{notoccite}
\usepackage{fancyhdr}
\usepackage[font=small,labelfont=bf]{caption}
\usepackage{xspace}
\usepackage{xcolor}
\usepackage[largesc,theoremfont]{newpxtext}
\usepackage{multirow}
\usepackage{booktabs}
%\usepackage[switch*,modulo]{lineno}
%\linenumbers
\usepackage{cleveref}
\usepackage{todonotes}
\usepackage{mathtools}
\usepackage{subfigure}

\pagestyle{fancy}
\lhead{}
\rhead{Practia }

\begin{document}

\renewcommand\Authfont{\fontsize{12}{14.4}\selectfont}
\renewcommand\Affilfont{\fontsize{9}{10.8}\itshape}
\renewcommand\Authand{,}
\renewcommand\Authands{,}

\title{Predicción de ventas de hamburguesas Krustyburger}

\author[1]{Alexis~Pacek}


\affil[1]{\footnotesize Practia Global, Buenos Aires, Argentina}

\date{Octubre 2021}

\maketitle

\begin{abstract}


\end{abstract}

\section{Exploración de los datos}



\begin{lstlisting}[language=Python, caption=Python example]
import numpy as np
    
def incmatrix(genl1,genl2):
    m = len(genl1)
    n = len(genl2)
    M = None #to become the incidence matrix
    VT = np.zeros((n*m,1), int)  #dummy variable
    
    #compute the bitwise xor matrix
    M1 = bitxormatrix(genl1)
    M2 = np.triu(bitxormatrix(genl2),1) 

    for i in range(m-1):
        for j in range(i+1, m):
            [r,c] = np.where(M2 == M1[i,j])
            for k in range(len(r)):
                VT[(i)*n + r[k]] = 1;
                VT[(i)*n + c[k]] = 1;
                VT[(j)*n + r[k]] = 1;
                VT[(j)*n + c[k]] = 1;
                
                if M is None:
                    M = np.copy(VT)
                else:
                    M = np.concatenate((M, VT), 1)
                
                VT = np.zeros((n*m,1), int)
    
    return M
\end{lstlisting}





\section{Conclusiones}

dsfdsgdfg


\begin{thebibliography}{99}

\bibitem{EnergySpectrum}
V.~Novotny for the Pierre Auger Coll. \textit{Energy spectrum of cosmic rays measured using the Pierre Auger Observatory}. Proc. of Science, 37$^\text{th}$ Int. Cosmic Ray Conf. (ICRC), 395, 324, 2021.

\bibitem{FDCalib}
The Pierre Auger Coll. \textit{Spectral Calibration of the Fluorescence Telescopes of the Pierre Auger Observatory}. Astropart. Phys., 95, 3, 2017.

\end{thebibliography}


\end{document}
